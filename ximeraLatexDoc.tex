\documentclass{article}
\usepackage{xcolor}
\usepackage{listings}
\usepackage{enumitem}
\usepackage[margin=1in]{geometry}

\newcommand\code[1]{\texttt{#1}}

\newcommand{\doc}[3]{\noindent \fbox{\code{#1}}
\begin{itemize}[align=left, itemsep=0pt, itemindent=10pt]
	\item[\textbf{Print:}] #2
	\item[\textbf{Online:}] #3
\end{itemize}}

\lstset{basicstyle=\normalsize\ttfamily, 
	language={[LaTeX]TeX},
	tabsize=2,
	keywordstyle=\color{blue!50!black},
	identifierstyle=\color{red!50!black},
	morekeywords={answer},
	breaklines=true}
	
\title{Using the \code{ximera} class}
\author{The Ximera Project}

\setlength{\parskip}{10pt}
\setlength{\parindent}{0pt}

\begin{document}

\maketitle

\section{Introduction}

The goal of the Ximera Project is to allow authors to write online interactive content though \LaTeX\ documents.  The \code{ximera} class provides the structure for a \LaTeX\ document to be displayed in a dynamic online environment.

Each \code{.tex} document written in the \code{ximera} class is an \emph{activity}.  Generally, an activity will be composed of \emph{content} and \emph{problems}.  Content is material served to the student, while problems require a response from the student.  The basic structure of an activity is:

\begin{lstlisting}
\documentclass{ximera}
\title{Our Sample Activity}
\begin{document}
\begin{abstract}
A sample document.
\end{abstract}
\maketitle

Let's multiply!
\begin{problem}
	What is $3 \times 2$? \answer{6}
\end{problem}
\end{document}
\end{lstlisting}
Note that every \code{ximera} document must have a \code{title}.

\section{Class Keys}

\doc{handout}
{hides the contents of \code{prompt}, \code{hints}, and \code{answer}}
{no effect.}


\doc{space}
{adds space for student work after problems}
{no effect.}


\doc{newpage}
{places each new problem on a new page}
{no effect.}

\section{Content}

To include an image, use the \code{image} environment.
\begin{lstlisting}
\begin{image}
	\includegraphics{myPrettyPicture.png}
\end{image}
\end{lstlisting}

The image environment is simply a generic way to indicate that an
image is present, and mostly deals with spaceing within the
document. When including basic TikZ images, please ensure they do not
rely on information outside of the \code{tikzpicture} environment.
\begin{lstlisting}
\begin{image}
	\begin{tikzpicture}
		TikZ code goes here.
	\end{tikzpicture}
\end{image}
\end{lstlisting}

To include a YouTube video, use the \code{youtube} command.  For all other videos, use the \code{video} command.
\begin{lstlisting}
\youtube{some-youtube-url}
\video{some-video-url}
\end{lstlisting}


There is support for interactive javascript content via the \code{interactive} environment:
\begin{lstlisting}
\begin{interactive}[interactiveContent.js]
	Some static content.
\end{interactive}
\end{lstlisting}
This is not currently supported on the Ximera server. 


\section{Problems and Answers}

A \code{problem} environment collects at least one \code{answer} from the student.
\begin{lstlisting}
\begin{problem}
	What is $3 \times 2$? \answer{6}
\end{problem}
\end{lstlisting}
Online this displays the problem ``What is $3 \times 2$?'' followed by an answer box.  The answer box will accept any answer that evaluates to 6, such as \fbox{$12 - 6$}.

A \code{prompt} environment helps to give an online answer structure.  For example,
\begin{lstlisting}
\begin{problem}
	If Jack earned 12 dollars and spent 3 dollars, how many dollars does Jack have left?
	\begin{prompt}
		Jack has \answer{9} dollars left.
	\end{prompt}
\end{problem}
\end{lstlisting}
The \code{prompt} environment is hidden in \code{handout} mode.


\subsection{Problems with Multiple Parts}

A \code{problem} can contain multiple \code{answer} commands, which must all be correct for the Ximera server to consider the problem complete.

\begin{lstlisting}
\begin{problem}
	There are \answer{60} minutes in an hour, \answer{24} hours in a day, 
		and \answer{7} days in a week.
\end{problem}
\end{lstlisting}

To display parts of a problem in a list, use \code{parts}.
\begin{lstlisting}
\begin{problem}
	\begin{parts}
		\item How many minutes in an hour? \answer{60}
		\item How many hours in a day? \answer{24}
		\item How many days in a week? \answer{7}
	\end{parts}
\end{problem}
\end{lstlisting}

To force an online student to complete the parts sequentially, add the \code{hide} key to \code{parts}.  Initially only the first \code{item} will be shown.  A correct submission to all answers in an \code{item} reveals the next one.  The problem is considered complete only when all answers have been correctly submitted.

\begin{lstlisting}
\begin{problem}
	\begin{parts}[hide]
		\item How many minutes in an hour? \answer{60}
		\item How many hours in a day? \answer{24}
		\item How many days in a week? \answer{7}
	\end{parts}
\end{problem}
\end{lstlisting}


\section{Hints}

A \code{hints} environment is a list of hints.
\begin{lstlisting}
\begin{problem}
	What is $3 \times 2$? \answer{6}
	\begin{hints}
		\item Try multiplying.
		\item Try 6.
	\end{hints}
\end{problem}
\end{lstlisting}
Online this adds a ``Show Hint'' button to the problem, which reveals the list of hints one at a time.

\end{document}
