%% You can put user macros here at this point
%% However, you cannot make new environments
\documentclass{ximera}
\usepackage{amssymb}
\usepackage{amsthm}
\usepackage{array}
\usepackage{amsmath}
\title{Testing Document}
\begin{document}
\maketitle
\section{Geogebra}
\geogebra{hZJZZzok}{980}{760}
\begin{problem}
\begin{multipleChoice}
What year was CSU founded?
\choice{1863}
\choice{1868}
\choice[correct]{1870}
\choice{1880}
\end{multipleChoice}
\end{problem}
\begin{problem}
Do the following tasks and then then answer the question.
    -Move points A and B such that they are 4 units apart.
    -Move point C to create a right angle at \HCode{<u>/</u>}B and where C is 3 units from B
    
Question:What is the measure of segment AC?
$\answer{5}$
\begin{hint}
Use Pythagorean Theorem
\end{hint}
\begin{problem}
Solve for x
 \[\frac{x-4}{2}=\frac{x+3}{4}\]
$\answer{5}$
\begin{hint}
Cross multiply
\end{hint}
\end{problem}
\end{problem}
 
 
\HCode{<iframe src="https://www.desmos.com/calculator/h3l7851caf" width="960px" height="780px"> </iframe>}
\youtube{75xO9xy7TTQ}
  
  
\HCode{<a rel="license" href="http://creativecommons.org/licenses/by-nc-sa/4.0/"><img alt="Creative Commons License" style="border-width:0" src="https://i.creativecommons.org/l/by-nc-sa/4.0/88x31.png" /></a><br />This work by <span xmlns:cc="http://creativecommons.org/ns#" property="cc:attributionName">Stan Kruse</span> is licensed under a <a rel="license" href="http://creativecommons.org/licenses/by-nc-sa/4.0/">Creative Commons Attribution-NonCommercial-ShareAlike 4.0 International License</a>.}
\noindent  {\Large\underline{\textbf{Problem:}}}
\noindent \ Find all integer solutions to the equation
$$ \frac{1}{x} + \frac{1}{y} = \frac{1}{z}. $$ \\ \\

\noindent  {\Large\underline{\textbf{Solution:}}}
\noindent \ We will show that all solutions are of the form:
$$ \begin{array}{l}
x = k \cdot a \cdot \left(a + b\right) \\
y = k \cdot b \cdot \left(a + b\right) \\
z = k \cdot a \cdot b,
\end{array} $$

\noindent where $k$, $a$, and $b$ are arbitrary non-zero integers, $a+b\neq0$. \\

\noindent  We can rewrite the given equation as
$$ z = \frac{xy}{x+y}, $$
so it suffices to find all pairs of integers $ x, y $ such that $x+y \mid xy$. \\
 
\noindent  \underline{\textbf{Lemma:}} \ Let $r$ and $s$ be relatively prime positive integers.  Then $r\pm s$ and $rs$ are relatively prime. \\

\noindent  \underline{\textbf{Proof:}} \ Suppose not; then there exists an integer $k > 1$ such that $k \mid r\pm s$ and $k \mid rs$.  We have \\  

\noindent  \[ k \mid r \pm s \Rightarrow k \mid r^2 \pm rs \Rightarrow k \mid r^2. \]
Similarly, $ k \mid s^2. $

Let $p > 1$ be any prime factor of $k$, so that $p \mid r^2$ and $p \mid s^2$.  
Now simply note that for any positive integer $m$, $ p \nmid m \Rightarrow p \nmid m^2$. 
Hence, we must have $p \mid r$ and $p \mid s$, which contradicts our assumption that $r$ and $s$ are relatively prime.  So $r \pm s$ and $rs$ must be relatively prime. // \\

Let $x$ and $y$ be integers such that $x+y \mid xy$.  We clearly cannot have $x+y = 0$.
If $\left|x+y\right| = 1$, then we have the solution sets
$$ \begin{array}{l}
x = c \\
y = -\left(c + 1\right) \\
z = c\cdot\left(c+1\right)
\end{array} $$
and
$$ \begin{array}{l}
x = c+1 \\
y = -c \\
z = -c\cdot\left(c+1\right),
\end{array} $$
where $c$ is an integer, $c \neq 0, 1$.

We now assume $\left|x+y\right| > 1 $.  
Let $n = \gcd\left(\left|x\right|,\left|y\right|\right)$.  Suppose that $\left|x\right|$ and $\left|y\right|$ are relatively prime.  
Then by the lemma, $\left|x+y\right|$ and $\left|xy\right|$ are relatively prime, a contradiction, since $x+y \mid xy$ and $\left|x+y\right| > 1 $.

\noindent Hence, $n > 1$.  Choose integers $a, b$ such that $x = na$ and $y = nb$.
This gives:
$$ \begin{array}{l}
\ \  \ \left(na + nb\right) \mid \left(na\right) \cdot \left(nb\right) \\ 
\Leftrightarrow n \cdot \left(a+b\right) \mid n^2 \cdot a \cdot b \\ 
\Leftrightarrow a+b \mid n \cdot a \cdot b. 
\end{array} $$
Since $n = \gcd\left(\left|x\right|,\left|y\right|\right)$, $\left|a\right|$ and $\left|b\right|$ must be relatively prime.
So by the lemma, the above holds if and only if $a+b \mid n$.  Put $n = k \cdot \left(a + b\right)$.  This gives the solution set:

$$ \begin{array}{l}
x = k \cdot a \cdot \left(a + b\right) \\
y = k \cdot b \cdot \left(a + b\right) \\
z = k \cdot a \cdot b.
\end{array} $$

\noindent Finally, note that the previous two solution sets are contained within this one (for the first set, take $k = -1, \ a = c, \ b = -\left(c+1\right)$; for the second set, take $k = 1, \ a = c+1, \ b= -c$).  Hence, this is the entire family of solutions, as desired.
\noindent   \hspace{\stretch{1}} $\blacksquare$ \\
\end{document}
