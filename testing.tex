%% You can put user macros here at this point
%% However, you cannot make new environments
\begin{Verbatim}
\documentclass{ximera}
\title{Testing Document}
\begin{document}
\begin{abstract}
The purpose of this file is to test features, create activities, and use Ximera funcionality.
\end{abstract}
\maketitle
\begin{question}
What is the capital of Colorado?
\begin{solution}
Denver /answer{$6$}
\end{solution}
\end{question}
%2
\begin{question} 
Suppose you are standing on a bridge that is 60 meters above
sea-level. You toss a ball up into the air with an initial velocity of
30 meters per second.  If $t$ is the time (in seconds) after we toss
the ball, then the height at time $t$ is approximately $f(t) = -5 t^2
+30t+60$. What does $f(2)$ mean in our context?
\begin{solution}
\begin{hint}
We want an answer in the context of the problem. 
\end{hint}
\begin{freeResponse}
The value $f(2)$ is the height of the ball after $2$ seconds.
\end{freeResponse}
\end{solution}
Now suppose $t$ is such that $f(t) = 100$. What does this mean in our
context?
\begin{solution}
\begin{hint}
We want an answer in the context of the problem. 
\end{hint}
\begin{freeResponse}
These value of $t$ are the times when the ball is at 100 meters above sea level.\end{freeResponse}
\end{solution}
Finally, if $h$ is a small positive value what is the meaning of
$f(t+h)$? How does this compare to the meaning of $f(t)+h$?
\begin{solution}
\begin{hint}
We want an answer in the context of the problem. 
\end{hint}
\begin{freeResponse}
The value $f(t+h)$ gives the height of the ball slightly after time
$t$. On the other hand, the value $f(t)+h$ gives a height just higher
than the ball at time $t$.
\end{freeResponse}
\end{solution}
\end{question}
%3
\begin{question} 
Suppose you are standing on a bridge that is 60 meters above
sea-level. You toss a ball up into the air with an initial velocity of
30 meters per second.  If $t$ is the time (in seconds) after we toss
the ball, then the height at time $t$ is approximately $f(t) = -5 t^2
+30t+60$. What does $f(2)$ mean in our context?
\begin{solution}
\begin{hint}
We want an answer in the context of the problem. 
\end{hint}
\begin{freeResponse}
The value $f(2)$ is the height of the ball after $2$ seconds.
\end{freeResponse}
\end{solution}
Now suppose $t$ is such that $f(t) = 100$. What does this mean in our
context?
\begin{solution}
\begin{hint}
We want an answer in the context of the problem. 
\end{hint}
\begin{freeResponse}
These value of $t$ are the times when the ball is at 100 meters above sea level.\end{freeResponse}
\end{solution}
Finally, if $h$ is a small positive value what is the meaning of
$f(t+h)$? How does this compare to the meaning of $f(t)+h$?
\begin{solution}
\begin{hint}
We want an answer in the context of the problem. 
\end{hint}
\begin{freeResponse}
The value $f(t+h)$ gives the height of the ball slightly after time
$t$. On the other hand, the value $f(t)+h$ gives a height just higher
than the ball at time $t$.
\end{freeResponse}
\end{solution}
\end{question}
\HCode{<iframe scrolling="no" src="https://tube.geogebra.org/material/iframe/id/hZJZZzok/width/800/height/600/border/888888/rc/false/ai/false/sdz/true/smb/false/stb/false/stbh/true/ld/false/sri/true/at/auto" width="600px" height="400px" style="border:0px;"> </iframe>}
\begin{question}
Do the following tasks and then then answer the questions.
Move points A and B such that they are 3 units apart.
Move point C to create a right angle at \HCode{<u>/</u>}B and where C is 4 units from B
What is the measure of segment AC?
\begin{solution}
5
\end{solution}
\end{question}
\end{document}
\end{Verbatim}
